% !TEX TS-program = arara
% arara: xelatex
% arara: animate: {options: [-alpha, remove]}
% arara: indent: { overwrite: yes }
% arara: clean: {extensions: [aux, idx, ilg, ind, log, bbl, bcf, ist, blg, run.xml, nlo, nls, out, synctex.gz]}

\documentclass[tikz]{standalone}

\usepackage{amssymb,amsmath}
\usepackage{upgreek} % upright greek symbols

% xetex
\usepackage{mathspec}
%\usepackage[no-math]{fontspec}
\usepackage{xltxtra}
\usepackage{xunicode}

\setmainfont[BoldFont={Fira Sans}]{Fira Sans Light}
\setmonofont{Fira Mono}

\setmathsfont(Digits,Latin,Greek){Fira Sans Light}
\setmathrm{Fira Sans Light}

% 中文字体
\usepackage{xeCJK}
    \setCJKmainfont[BoldFont={Hiragino Sans GB W6}, ItalicFont={Kaiti SC Regular}]{Hiragino Sans GB W3}
    \xeCJKsetup{CJKecglue = {\hskip 0pt plus 0.08\baselineskip}, xCJKecglue = {false}}
    \punctstyle{plain}

% don't move this line above
\defaultfontfeatures{Mapping=tex-text,Scale=MatchLowercase}

% use upquote if available, for straight quotes in verbatim environments
\IfFileExists{upquote.sty}{\usepackage{upquote}}{}

% use microtype if available
\IfFileExists{microtype.sty}{%
    \usepackage{microtype}
    \UseMicrotypeSet[protrusion]{basicmath} % disable protrusion for tt fonts
    }{}

% https://tex.stackexchange.com/questions/272790/xecjk-breaks-non-chinese-characters
\AtBeginDocument{%
  \XeTeXcharclass`^^^^2026=0
  \XeTeXcharclass`^^^^2019=0
}

\usepackage{color}
\usepackage{animate}

\usepackage{tikz}
\usepackage{pgfplots}
\usetikzlibrary{
    calc,
    arrows,
    positioning
}

% Paul Tol's 12-color palette
\definecolor{TolDarkBlue}{HTML}{6699CC}
\definecolor{TolDarkBrown}{HTML}{999933}
\definecolor{TolDarkGreen}{HTML}{117733}
\definecolor{TolDarkPink}{HTML}{882255}
\definecolor{TolDarkPurple}{HTML}{332288}
\definecolor{TolDarkRed}{HTML}{661100}
\definecolor{TolLightBlue}{HTML}{88CCEE}
\definecolor{TolLightBrown}{HTML}{DDCC77}
\definecolor{TolLightGreen}{HTML}{44AA99}
\definecolor{TolLightPink}{HTML}{AA4466}
\definecolor{TolLightPurple}{HTML}{AA4499}
\definecolor{TolLightRed}{HTML}{CC6677}


\begin{document}

% https://gist.github.com/seungwonpark/b10d86dd27920bba4b33d41e81855c47
% https://en.wikipedia.org/wiki/File:Fourier_series_square_wave_circles_animation.gif

\foreach \angle in {0,4,...,360}{
    \begin{tikzpicture}
        \node at (0,2.5) {\fbox{\parbox{6cm}{
                    \centering
                    $ \displaystyle f(x) = \frac{4}{\pi} \sum_{2\nmid n}^{\infty}
                        \frac{1}{n}\sin\left(\frac{n\pi x}{\mathrm{L}}\right) $
                }}};

        \begin{scope}[shift={(0,0)}]
            % wave
            \draw[TolDarkBlue]
            plot[smooth,samples=100,domain=\angle-360:\angle]
            (pi/180*\angle-pi/180*\x,{sin(\x)});

            % epicycles
            \coordinate (O1) at (-2,0);
            \draw[TolDarkBlue] (O1) circle [radius=1];
            \draw[TolDarkBlue] (O1) -- +(\angle:1) coordinate (E1);

            \begin{scope}[shift={(-2.5,0)}]
                \coordinate (O) at (-2,0);
                \draw[TolDarkBlue] (O) circle [radius=1];
                \draw[TolDarkBlue] (O) -- +(\angle:1);
            \end{scope}

            % link
            \filldraw
            (E1) circle (1pt) --
            (0,{sin(\angle)}) circle (1pt);

            \node at (-2,-1.3) {\footnotesize $ (n=1) $};
        \end{scope}

        \begin{scope}[shift={(0,-3)}]
            % wave
            \draw[TolDarkBrown]
            plot[smooth,samples=100,domain=\angle-360:\angle]
            (pi/180*\angle-pi/180*\x,{sin(\x)+1/3*sin(3*\x)});

            % epicycles
            \coordinate (O1) at (-2,0);
            \draw[TolDarkBlue] (O1) circle [radius=1];
            \draw[TolDarkBlue] (O1) -- +(\angle:1) coordinate (E1);

            \coordinate (O3) at (E1);
            \draw[TolDarkBrown] (O3) circle [radius=1/3];
            \draw[TolDarkBrown] (O3) -- +(3*\angle:1/3) coordinate (E3);

            \begin{scope}[shift={(-2.5,0)}]
                \coordinate (O) at (-2,0);
                \draw[TolDarkBrown] (O) circle [radius=1/3];
                \draw[TolDarkBrown] (O) -- +(3*\angle:1/3);
            \end{scope}

            % link
            \filldraw
            (E3) circle (1pt) --
            (0,{sin(\angle)+1/3*sin(3*\angle)}) circle (1pt);

            \node at (-2,-1.3) {\footnotesize $ (n=1,3) $};
        \end{scope}

        \begin{scope}[shift={(0,-6)}]
            % wave
            \draw[TolDarkGreen]
            plot[smooth,samples=100,domain=\angle-360:\angle]
            (pi/180*\angle-pi/180*\x,{sin(\x)+1/3*sin(3*\x)+1/5*sin(5*\x)});

            % epicycles
            \coordinate (O1) at (-2,0);
            \draw[TolDarkBlue] (O1) circle [radius=1];
            \draw[TolDarkBlue] (O1) -- +(\angle:1) coordinate (E1);

            \coordinate (O3) at (E1);
            \draw[TolDarkBrown] (O3) circle [radius=1/3];
            \draw[TolDarkBrown] (O3) -- +(3*\angle:1/3) coordinate (E3);

            \coordinate (O5) at (E3);
            \draw[TolDarkGreen] (O5) circle [radius=1/5];
            \draw[TolDarkGreen] (O5) -- +(5*\angle:1/5) coordinate (E5);

            \begin{scope}[shift={(-2.5,0)}]
                \coordinate (O) at (-2,0);
                \draw[TolDarkGreen] (O) circle [radius=1/5];
                \draw[TolDarkGreen] (O) -- +(5*\angle:1/5);
            \end{scope}

            % link
            \filldraw
            (E5) circle (1pt) --
            (0,{sin(\angle)+1/3*sin(3*\angle)+1/5*sin(5*\angle)}) circle (1pt);

            \node at (-2,-1.3) {\footnotesize $ (n=1,3,5) $};
        \end{scope}

        \begin{scope}[shift={(0,-9)}]
            % wave
            \draw[TolDarkPink]
            plot[smooth,samples=100,domain=\angle-360:\angle]
            (pi/180*\angle-pi/180*\x,{sin(\x)+1/3*sin(3*\x)+1/5*sin(5*\x)+1/7*sin(7*\x)});

            % epicycles
            \coordinate (O1) at (-2,0);
            \draw[TolDarkBlue] (O1) circle [radius=1];
            \draw[TolDarkBlue] (O1) -- +(\angle:1) coordinate (E1);

            \coordinate (O3) at (E1);
            \draw[TolDarkBrown] (O3) circle [radius=1/3];
            \draw[TolDarkBrown] (O3) -- +(3*\angle:1/3) coordinate (E3);

            \coordinate (O5) at (E3);
            \draw[TolDarkGreen] (O5) circle [radius=1/5];
            \draw[TolDarkGreen] (O5) -- +(5*\angle:1/5) coordinate (E5);

            \coordinate (O7) at (E5);
            \draw[TolDarkPink] (O7) circle [radius=1/7];
            \draw[TolDarkPink] (O7) -- +(7*\angle:1/7) coordinate (E7);

            \begin{scope}[shift={(-2.5,0)}]
                \coordinate (O) at (-2,0);
                \draw[TolDarkPink] (O) circle [radius=1/7];
                \draw[TolDarkPink] (O) -- +(7*\angle:1/7);
            \end{scope}

            % link
            \filldraw
            (E7) circle (1pt) --
            (0,{sin(\angle)+1/3*sin(3*\angle)+1/5*sin(5*\angle)+1/7*sin(7*\angle)}) circle (1pt);

            \node at (-2,-1.3) {\footnotesize $ (n=1,3,5,7) $};
        \end{scope}

    \end{tikzpicture}
}

\end{document}
